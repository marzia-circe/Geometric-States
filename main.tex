% ****** Start of file apssamp.tex ******
%
%   This file is part of the APS files in the REVTeX 4.1 distribution.
%   Version 4.1r of REVTeX, August 2010
%
%   Copyright (c) 2009, 2010 The American Physical Society.
%
%   See the REVTeX 4 README file for restrictions and more information.
%
% TeX'ing this file requires that you have AMS-LaTeX 2.0 installed
% as well as the rest of the prerequisites for REVTeX 4.1
%
% See the REVTeX 4 README file
% It also requires running BibTeX. The commands are as follows:
%
%  1)  latex apssamp.tex
%  2)  bibtex apssamp
%  3)  latex apssamp.tex
%  4)  latex apssamp.tex
%
\documentclass[%
% reprint,
%superscriptaddress,
%groupedaddress,
%unsortedaddress,
%runinaddress,
%frontmatterverbose, 
%preprint,
%showpacs,preprintnumbers,
nofootinbib,
%nobibnotes,
%bibnotes,
 amsmath,amssymb,
aps,
%pra,
%prb,
%rmp,
%prstab,
%prstper,
%floatfix,
]{revtex4-1}

\usepackage{graphicx}% Include figure files
\usepackage{dcolumn}% Align table columns on decimal point
\usepackage{bm}% bold math
%\usepackage{hyperref}% add hypertext capabilities
%\usepackage[mathlines]{lineno}% Enable numbering of text and display math
%\linenumbers\relax % Commence numbering lines
\usepackage{subfigure}
\usepackage{amsthm}
\usepackage{amsmath}
%\usepackage{graphicx,caption,subcaption}
\graphicspath{{./Figs/}}
\usepackage{bbold}
\usepackage{tikz}
\usepackage{tkz-euclide}
\usepackage{adjustbox}
\usepackage{pst-solides3d}
\usetikzlibrary{matrix}
\usepackage{tikz-3dplot}
\usepackage[all,cmtip]{xy}

%\usepackage[showframe,%Uncomment any one of the following lines to test 
%%scale=0.7, marginratio={1:1, 2:3}, ignoreall,% default settings
%%text={7in,10in},centering,
%%margin=1.5in,
%%total={6.5in,8.75in}, top=1.2in, left=0.9in, includefoot,
%%height=10in,a5paper,hmargin={3cm,0.8in},
%]{geometry}
\theoremstyle{plain}% default
\newtheorem{thm}{Theorem}[section]
\newtheorem{lem}[thm]{Lemma}
\newtheorem{prop}[thm]{Proposition}
\theoremstyle{definition}
\newtheorem{defn}[thm]{Definition}
\newtheorem{exmp}[thm]{Example}
\theoremstyle{remark}
\newtheorem{rem}[thm]{Remark}
\usepackage{braket}


\def\nn{\nonumber}
\def\hom{\text{hom}(C,G)}
\def\homA{\text{hom}(C_A,G)}
\def\homtA{\text{hom}(C_{\tilde{A}},G)}
\def\homiA{\text{hom}(C_{\text{int}(A)},G)}
\def\homB{\text{hom}(C_B,G)}
\def\Hom{\text{Hom}}
\def\Z{\mathbb{Z}}
\def\H{\mathcal{H}}
\def\C{\mathbb{C}}

\def\ker{\text{ker}}
\def\im{\text{Im}}

\begin{document}

%\preprint{APS/123-QED}

\title{Ground State Degeneracy from Geometry in 3D}% Force line breaks with \\
%\thanks{A footnote to the article title}%

\author{J. P. Ibieta-Jimenez}
\email{pibieta@if.usp.br}

\author{M. Petrucci}%
 \email{marzia@if.usp.br}
 %\altaffiliation[Also at ]{Physics Department, XYZ University.}%Lines break automatically or can be forced with \\
\author{P. Teotonio-Sobrinho}%
 \email{teotonio@if.usp.br}
\affiliation{%
 Departamento de F\'isica Matem\'atica, Universidade de S\~ao Paulo\\ Rua do Mat\~ao Travessa R 187, CEP 05508-090, S\~ao Paulo, Brazil.
}%

\date{\today}% It is always \today, today,
             %  but any date may be explicitly specified

\begin{abstract}
We need an abstract....
\begin{description}
\item[Usage]
Secondary publications and information retrieval purposes.
\item[PACS numbers]
May be entered using the \verb+\pacs{#1}+ command.
\item[Structure]
You may use the \texttt{description} environment to structure your abstract;
use the optional argument of the \verb+\item+ command to give the category of each item. 
\end{description}
\end{abstract}

\pacs{Valid PACS appear here}% PACS, the Physics and Astronomy
                             % Classification Scheme.
%\keywords{Suggested keywords}%Use showkeys class option if keyword
                              %display desired
\maketitle



\section{\label{sec:intro}Introduction}
\begin{itemize}
\item Topological Order
\item Fractons
\item extensive degeneracy
\end{itemize}
The formalism developed in \cite{pramod} allows for the obtention of a large class of models where the bosonic degrees of freedom at vertices are coupled with gauge fields at links. Many such models can be got by the mere choice of the main ingredients, i.e., the gauge group $G$ and a $\mathbb{C}[G]$-module $(V_n, \mu)$. Aiming at a higher gauge theory interpretation, such models can be understood as a $0,1$-gauge theory. Since degrees of freedom are associated to both $0-cells$ (vertices) and $1-cells$ (links). In \cite{kazuothesis,ricardo} a very similar formalism is constructed for (3+1) topological phases using a $2$-group as the gauge group. This is, a $1,2$-gauge theory for topological phases of matter was constructed. In this case, degrees of freedom are associated to the $1-cells$ (links) and $2-cells$ (plaquettes). Taking these ideas into consideration, in \cite{higher} we constructed a Hamiltonian model for topological phases of matter in arbitrary dimensions. Furthermore, the notion of gauge group is promoted to a more general mathematical structure, that of a \textit{chain complex of abelian groups}. In this new framework, it can be shown \cite{higher} that the ground state degeneracy is a topological invariant of an abstract co-chain complex. In this sense, the ground state of such models is completely characterized and understood from the $0$-th cohomology group associated to such co-chain complex. Moreover, a theorem due to Brown \cite{Brown} relates this rather abstract cohomology group with the usual cohomology groups, enabling for a natural geometrical interpretation of the ground state subspace. 

Recently, we have found that by considering some of these fields as having only \emph{classical configurations}, the otherwise purely topological models now become sensitive to the geometry of the manifold upon which they are defined. This is showcased, as in the purely topological case, in the ground states of the Hamiltonian models, as we will show in the next section for a particular example. This fact has turned our attention to these new class of models that, we believe, are related to some other new findings in the topological phases of matter community. In particular, there is a strong interest in a new class of quantum states called \emph{fracton} phases \cite{chamon05,bravyi11,castelnovo12,haah11,bravyi13,yoshida13,vijay15,vijay16,vijay17,shirley17,rahul18}. Some characteristic of fracton phases include elementary excitations with restricted mobility and in some cases a ground state degeneracy that explicitly depends on the geometry of the manifold \cite{slagle17}. The latter has caught our attention as the higher gauge theories constructed in \cite{higher} can be modified to obtain a source of models with a geometrical dependence of the ground state degeneracy. The systematic construction of such models consists on restricting some $l$-constituents of the $n$-gauge configurations to be classical; This leads to equivalence classes of Hamiltonian models, some of which, have a geometrical dependence of the ground state degeneracy.  The construction of the models starting from the full higher gauge theories, the analysis of the unitary classes of Hamiltonians and the study of the geometrical models constitute the main contents of the Ph.D. research project. In this sense, the main objective of the research project is to characterize the topological and geometrical phases coming from the procedure of partially freezing the degrees of freedom. This endeavor is partially achieved since the construction of the models and their characterization with respect to the appearance of geometrical phases or not is known, as we will show in the next section. There are, however, some details that have to be concealed before considering the project as finished, they include:
\begin{itemize}
\item A Proof that the geometrical states are indeed the ground states of such Hamiltonians.
\item A brief study of the excited states, paying special attention to the mobility restrictions that appear in fracton phases.
\item To fit the description of the models into the formalism developed in \cite{higher}.
\end{itemize}

\section{Quantum Models }\label{sec:QModels}
\subsection{2D Geometric Quantum Model }\label{sec:2Dmodel}
\subsection{3D Geometric Quantum Model }\label{sec:3Dmodel}

 
\section{Geometric Ground State Degeneracy}\label{sec:G-GSD}
\subsection{2D Model}
\subsection{3D Model}

\section{Conclusions}\label{sec:Remarks}

\appendix






\section{Trace of Local Operators}\label{sec:app2}
In this appendix we show how taking the partial trace of the ground state projector, or any product of projection operators of the theory, implies in Eq.(\ref{eq:rhoA1}). This is, we show that the local projectors \(A_{n,x}\) and \(B_{n,x}\) are traceless unless they are trivial (equal to the identity operator).

We begin by writing the density matrix, \(\rho\), using the local decomposition of \(\mathcal{A}_0\) and \(\mathcal{B}_0\) (see \cite{higher} for a detailed account on this). The local decomposition yields:
\begin{align*}
\mathcal{A}_0 = \prod_{n=0}^{d} \prod_{x\in K_n} A_{x}, \; \text{and}\quad \mathcal{B}_0 = \prod_{n=0}^{d} \prod_{x\in K_n} B_{x},
\end{align*}
such that the density matrix of Eq.(\ref{eq:rho}) can be written as:
\begin{align*}
\rho = \dfrac{1}{GSD} \left(\prod_{n=0}^{d} \prod_{x\in K_n} A_{x}\right)\left(\prod_{n=0}^{d} \prod_{x\in K_n} B_{x}\right),
\end{align*}
this form is convenient for taking the partial trace as the operators are now localized at simplices $x \in K_n$ for $0\leq n \leq d$, this allows for the identification of the operators that act exclusively on region $A$ from the operators that act on both $\partial(A)$ and $B$, in order to get the terms that survive the partial trace.
In this sense, the reduced density matrix is written as:
\begin{align}\label{eq:rhoAlocal}
\rho_A = \text{Tr}_B(\rho)= \;\text{Tr}_B\left(\prod_n \prod_{x \in K_n}A_{x}\prod_{y \in K_n}B_{y}\right).
\end{align}
Before going onto the calculation of the above partial trace, we will prove a property that will let us evaluate the partial trace rather straightforwardly.


\begin{prop}\label{prop:traceless}
Let \(x,y \in K_n\), be $n$-simplices for \(0\leq n\leq d\). The local operators, \(A_{x}, B_{y}:\mathcal{H}\rightarrow\mathcal{H}\), are traceless unless they act  trivially (as the identity operator \(\mathbb{1}_\mathcal{H}\)).
\end{prop}
\begin{proof}
Let \(\{\ket{f}\}\) be a basis of $\mathcal{H}$, with \(f \in \hom^0\). We start by taking the trace of the local gauge transformations:
\begin{align*}
\text{Tr}\left(A_{x}\right) = & \sum_{f} \bra{f} A_{x} \ket{f} = \dfrac{1}{\left|G_{n+1}\right|}\sum_{f}\, \sum_{g \in G_{n+1}} \bra{f} A_{e[x,g]} \ket{f}.
\end{align*}
From Definition \ref{def:AopBop}, the action of \(A_{n,x}\) on a basis state consists in general on a shift of basis elements, which yields:
\begin{align*}
\text{Tr}\left(A_{x}\right) =&  \dfrac{1}{\left|G_{n+1}\right|}\sum_{f}\, \sum_{g \in G_{n+1}} \bra{f}  \ket{f+ \delta^{-1}(e[x,g])}.
\end{align*}
From the last expression it is clear that the only non-null term in the sum occurs only when \(g=e \in G_{n+1}\), the identity element. Thus, we have:
\begin{align*}
\text{Tr}\left(A_{n,x}\right) =& \dfrac{\text{Tr}\left(\mathbb{1}\right)}{\left|G_{n+1}\right|} = \dfrac{\dim(\mathcal{H})}{\left|G_{n+1}\right|}.
\end{align*}
 Similarly, for the trace of local holonomy measurement operators,\(B_{n,y}\), we have:
\begin{align*}
\text{Tr}\left(B_{y}\right) = & \sum_{f} \bra{f} B_{x} \ket{f} = \dfrac{1}{\left|G_{n-1}\right|}\sum_{f}\, \sum_{r \in \hat{G}_{n-1}} \bra{f} B_{\hat{e}[y,r]} \ket{f}. 
\end{align*}
Using Definition \ref{def:AopBop} the above expression can be written as:
\begin{align*}
\text{Tr}\left(B_{y}\right) =\;  \dfrac{1}{\left|G_{n-1}\right|}\sum_{f}\, \sum_{r \in \hat{G}_{n-1}} \chi_r\left({f_n(y)}\right)\braket{f|f}\;
=\;  \dfrac{1}{\left|G_{n-1}\right|}\sum_{f}\, \sum_{r \in \hat{G}_{n-1}} \chi_r\left({f_n(y)}\right)\chi_{\hat{e}}\left({f_n(y)}\right)\braket{f|f},
\end{align*}
where in the last line we used the fact that \(\chi_{\hat{e}}(g)=1, \, \forall g \in G_{n-1}\) and \(\hat{e} \in \hat{G}_{n-1}\), the trivial representation. From the orthogonality relations of characters \cite{hall, barut,serre, james}, we note that:
\begin{align*}
\sum_{f}\chi_r\left({f_n(y)}\right)\chi_e\left({f_n(y)}\right) = \delta(e,f_n(y)), 
\end{align*}
which implies that the trivial representation term is the only one that has non-zero trace, since it acts as the identity operator.
\begin{align*}
\text{Tr}\left(B_{y}\right) = & \dfrac{\left|\mathcal{H}\right|}{\left|G_{n-1}\right|}
\end{align*}
\end{proof}
 This result can naturally be extended to products of such operators to show that the only term that survives the trace is the one that acts trivially. This allows us to express  the reduced density matrix, \(\rho_A\) of Eq.(\ref{eq:rhoAlocal}) in terms of operators that act only in region \(A\). 
 
 In this case,  Proposition \ref{prop:traceless} implies that any operator (or product of several) that is different from \(\mathbb{1}_B\), the identity operator in \(\mathcal{H}_B\), will have vanishing trace. This, in turn, tells us about the nature of the operators that survive the trace; In particular, local gauge transformations \(A_{x}\) will survive the trace if and only if \(x \in K_{n,\tilde{A}}\), where \(\tilde{A}\) is the interior of \(A\) \footnote{Local gauge transformations are labeled by simplices \(x \in K_n\) and they act on the gauge fields at the co-boundary, \(\partial^{\ast}(x)\). In particular, gauge transformations located at \(x \in K_{n,{\partial(A)}}\), the boundary of \(A\), also act on \(B\). Thus, they do not contribute to the trace.} as in Def. \ref{def:intA}. On the other hand, local holonomy measurement operators \(B_{y}\) will survive the trace if and only if \(y \in K_{n,A}\) which corresponds to the entire region $A$. Consequently, the reduced density matrix is:
\begin{align*}
\rho_A =\text{Tr}_B(\mathbb{1}_B) \prod_{n} \prod_{x \in K_{n,\tilde{A}}}A_{x}\prod_{y \in K_{n,A}}B_{y} .
\end{align*}  
From which we write Eq. (\ref{eq:rhoA1}).

\bibliographystyle{apsrev4-1}
\bibliography{bib.bib}
\end{document}
%
% ****** End of file apssamp.tex ******
